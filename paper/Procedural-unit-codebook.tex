% Options for packages loaded elsewhere
\PassOptionsToPackage{unicode}{hyperref}
\PassOptionsToPackage{hyphens}{url}
%
\documentclass[
]{article}
\usepackage{amsmath,amssymb}
\usepackage{iftex}
\ifPDFTeX
  \usepackage[T1]{fontenc}
  \usepackage[utf8]{inputenc}
  \usepackage{textcomp} % provide euro and other symbols
\else % if luatex or xetex
  \usepackage{unicode-math} % this also loads fontspec
  \defaultfontfeatures{Scale=MatchLowercase}
  \defaultfontfeatures[\rmfamily]{Ligatures=TeX,Scale=1}
\fi
\usepackage{lmodern}
\ifPDFTeX\else
  % xetex/luatex font selection
\fi
% Use upquote if available, for straight quotes in verbatim environments
\IfFileExists{upquote.sty}{\usepackage{upquote}}{}
\IfFileExists{microtype.sty}{% use microtype if available
  \usepackage[]{microtype}
  \UseMicrotypeSet[protrusion]{basicmath} % disable protrusion for tt fonts
}{}
\makeatletter
\@ifundefined{KOMAClassName}{% if non-KOMA class
  \IfFileExists{parskip.sty}{%
    \usepackage{parskip}
  }{% else
    \setlength{\parindent}{0pt}
    \setlength{\parskip}{6pt plus 2pt minus 1pt}}
}{% if KOMA class
  \KOMAoptions{parskip=half}}
\makeatother
\usepackage{xcolor}
\usepackage[margin=1in]{geometry}
\usepackage{graphicx}
\makeatletter
\def\maxwidth{\ifdim\Gin@nat@width>\linewidth\linewidth\else\Gin@nat@width\fi}
\def\maxheight{\ifdim\Gin@nat@height>\textheight\textheight\else\Gin@nat@height\fi}
\makeatother
% Scale images if necessary, so that they will not overflow the page
% margins by default, and it is still possible to overwrite the defaults
% using explicit options in \includegraphics[width, height, ...]{}
\setkeys{Gin}{width=\maxwidth,height=\maxheight,keepaspectratio}
% Set default figure placement to htbp
\makeatletter
\def\fps@figure{htbp}
\makeatother
\setlength{\emergencystretch}{3em} % prevent overfull lines
\providecommand{\tightlist}{%
  \setlength{\itemsep}{0pt}\setlength{\parskip}{0pt}}
\setcounter{secnumdepth}{5}
\usepackage{float}
\floatplacement{figure}{hp}
\usepackage{booktabs}
\usepackage[font=small,labelfont=bf]{caption}
\usepackage{setspace}\doublespacing
\ifLuaTeX
  \usepackage{selnolig}  % disable illegal ligatures
\fi
\IfFileExists{bookmark.sty}{\usepackage{bookmark}}{\usepackage{hyperref}}
\IfFileExists{xurl.sty}{\usepackage{xurl}}{} % add URL line breaks if available
\urlstyle{same}
\hypersetup{
  pdftitle={A codebook for summarizing presence or absence of stone tool-making techniques reported in the literature v.1.1},
  pdfauthor={-Jonathan Paige},
  pdfkeywords={keyword 1; keyword 2; keyword 3},
  hidelinks,
  pdfcreator={LaTeX via pandoc}}

\title{A codebook for summarizing presence or absence of stone
tool-making techniques reported in the literature v.1.1}
\author{-Jonathan Paige\footnote{University of Missouri,
  \href{mailto:jonathan.n.paige@gmail.com}{\nolinkurl{jonathan.n.paige@gmail.com}}}}
\date{}

\begin{document}
\maketitle

{
\setcounter{tocdepth}{3}
\tableofcontents
}
\hypertarget{version-details}{%
\section{Version details:}\label{version-details}}

This report was generated on 2024-04-06 18:58:20.13651 The current Git
commit details are:

\begin{verbatim}
## Local:    V.2.0-development C:/Users/Jon Paige/Desktop/Procedural.unit.codebook
## Remote:   V.2.0-development @ origin (https://github.com/jnpaige/Procedural.unit.codebook.git)
## Head:     [7b5a2aa] 2024-04-06: Updated how soft hammer, pressure flaking, and indirect percussion are coded.
\end{verbatim}

I made this codebook while working on ways of quantifying variation in
tool-making sequences across the archaeological record. In this case the
unit of analysis is the technique, or procedural unit (Perreault et al.,
2013). I coded procedural units as present or absent across
archaeological assemblages, based on published reports describing those
assemblages. Most systematic, comparative work on lithic technology has
focused on units of analysis that are more consistent, and replicable
across assemblages. For example flake and biface measurements, and shape
variation can be very consistent between studies. Other aspects of
lithic technology, like chaînes opératoire or reduction sequences that
we are interested in, are not as easy to analyze statistically.
Similarity and dissimilarity in them is difficult to quantify
systematically because they represent qualitative data conveyed through
prose descriptions, tables of artifact counts, and illustrations of
individual artifacts and schematics of those sequences. How sequences
are reported might vary by the research tradition of the analyst, and
how I interpret those reports is also shaped by the tradition I am a
part of. This is a big hurdle to making systematic comparisons of
technology across many assemblages (Reynolds, 2018). Reliably extracting
the presence or absence of procedural units from the complex
descriptions of stone tool technologies is a similar problem to that
faced by scientists who code interview transcripts, and other texts.
There are limits to human abilities to make sense of complex
information, and individual scientists may use their own heuristics, and
prior experience to collapse and make sense of that information. Those
heuristics, and mental shortcuts introduce bias and error into the
analysis (Hruschka et al., 2004; MacQueen et al., 1998; Tversky \&
Kahneman, 1974). One way to reduce this error is to give clear
definitions of each procedural unit. Clear definitions, however, are not
enough. If I gave a careful definition of the procedural unit ``core
tablet``, there may be many forms that could be accomodated into that
definition that I had not taken into account.

Likewise, other researchers may have different standards by which they
would count a`` core tablet`` as being present in an assemblage. Some
may be happy to find flakes that have the features of a core tablet, and
count them as core tablets. Others may only count flakes with those
features if there are also cores with negative removals consistent with
core tablet removals. Others may have very different ideas of what the
features of a core tablet are, and slightly different interpretations of
those features themselves. The problem with using only definitions, is
that it does not give enough guidance on what forms do not meet that
definition, and when we should code ``core tablet`` as present. Social
scientists at the Center for Disease Control encountered similar
problems while coding interview transcripts and other texts for
qualitative analysis. Their solution was to develop codebooks with
explicit inclusion, and exclusion criteria. This codebook follows a
format developed at the CDC to ensure reliability in coding texts
(MacQueen et al., 1998). Each procedural unit has a short, and longer
definition (if needed). Additionally, each also has inclusion criteria.
These inclusion criteria describe what features need to be present in a
report for us to count the procedural unit as present. Each also has
exclusion criteria. These describe what features in the text would be
grounds to count the procedural unit as absent. For example, many
procedural units only make sense to count as present if hierarchical
core reduction is present. So, one exclusion criteria for those
procedural units would be the absence of any evidence for hierarchical
core reduction. Each entry may also include typical, and atypical
examples of the procedural unit. These may be illustrations, or text
examples. These two help to further outline when it is appropriate to
code a procedural unit as present. Finally, there is a ``close, but no``
entry, which outlines what kinds of cases might be confused with the
procedural unit. Some of these inclusion criteria are more exhaustive
than others, especially in cases where I noticed that I came to
different conclusions about how a procedural unit should be coded
multiple times while double checking data. Whether or not you agree with
all the definitions, and coding criteria, you should be able to come
closer to the same conclusions about how to code the presence or absence
of any of these procedural units if you follow the standards outlined in
the codebook than would be the case without reporting the criteria I
employed. The codebook should evolve over time, as new logical
inconsistencies or points of confusion are discovered. However, whatever
version was used to code the data that then ends up in a final analysis,
should be archived and made available to readers. This version of the
codebook does not have example illustrations for procedural units to
avoid copyright issues, but references to the illustrations are
retained.

\hypertarget{v.1.1-notes}{%
\section{V.1.1 Notes}\label{v.1.1-notes}}

This modified version helps solve some issues with the first version.
The main issue is the lack of metadata in the dataset describing what
particular criteria where met in coding a procedural unit as present or
absent, and a citation referring to the figures, or paragraphs,
containing the relevant inclusion or exclusion criteria. Without that
metadata, it will be more challenging for others to assess the quality
of the data.

To help with this, the inclusion and exclusion criteria were tweaked to
make clear whether evidence in text, or evidence in figures should be
used as evidence either for inclusion or exclusion. This is then
mirrored in the procedural unit data sheet where additional columns
associated with each procedural unit record what inclusion/exclusion
criteria were met, and the page/paragraph and figure/table numbers that
meet those criteria.

Other changes include clearer guidance in most entries, removal of some
logical inconsistencies, folding in ``pressure flaking through retouch''
with ``pressure flaking'', removal of invasive retouch (folded in with
invasive flaking), and removal of bifacial retouch (folded in with
retouch)

\hypertarget{definitions}{%
\section{Definitions}\label{definitions}}

\textbf{Hierarchical cores} Cores where platforms are established to
strike flakes that shape the main flaking surface of a core, or the main
platform itself in preparation for removals across the main flaking
surface. In some cases these can be counted as present without figures
showing cores if there are unambiguous blanks from formal blade
technology, levallois reduction, microblade reduction, presence of
adzes, or other hierarchical reduction practices.

\textbf{Main flaking axis} On hierarchical cores, this is the axis
parallel to the face along which most target pieces were produced. On a
naviform core, for example, the main flaking axis is that along which
bidirectional blades are taken. Radial cores have no main flaking axis.
Centripetal levallois cores also have no main falking axis, while
preferential levallois cores do.

\textbf{Main flaking surface} On hierarchical cores, this is the surface
from which targetted blanks were removed.

\textbf{Core platform} Any of the edges percussed or applied with
pressure to remove flakes.

\textbf{Main core platform(s)} The platforms used to produce target
pieces. Blade platforms on blade cores. Facetted platform on a
preferential levallois.

\textbf{Core back} Posterior surface relative to the main flaking
surface, roughly parallel to the main flaking surface.

\textbf{Core bottom} Distal surface of a hierarchical core, opposite the
main core platform.

\hypertarget{coding-presenceabsence-of-hierarchical-cores}{%
\section{Coding presence/absence of hierarchical
cores}\label{coding-presenceabsence-of-hierarchical-cores}}

Since many of the procedural units require the presence of hierarchical
cores, or cores where there is an argued structure/plan involving
preparation of a core platform or face before targetted removals of
flakes, it is important to consistently code as present or absent
hierarchical flaking even if it is not treated as a technique on its
own.

\hypertarget{procedural-unit-codes}{%
\section{Procedural unit codes}\label{procedural-unit-codes}}

\hypertarget{raw-material-treatment}{%
\subsection{Raw material treatment}\label{raw-material-treatment}}

\textbf{Short Description:} Heat treatment of raw material

\textbf{Definition:} Heating of raw material in order to improve
workability. This process alters the fracture mechanics of raw material,
and often causes changes in texture, and color.

\textbf{Inclusion criteria:} Described in text.

\textbf{Exclusion criteria:} Not described in text.

\textbf{Typical exemplars:} ``Flint was/was likely heat treated''. Heat
treatment is described as present.

\textbf{Atypical exemplars:} Heating is described, and reference made to
glassy/glossy texture of raw material as a result of heating: ``flint
was glossed/waxy/greasy from heating''

\textbf{Close but no:} ``material was fired'', ``material bears signs of
thermal alteration'', Images of pieces that appear to have been heat
treated, but without accompanying text describing them as heat treated.
``Some flints had a glossy/waxy/greasy appearance''.

\hypertarget{faceting-of-core-platform}{%
\subsection{Faceting of core platform}\label{faceting-of-core-platform}}

\textbf{Short Description:} Shaping of a core platform by striking
flakes into the platform from the face of the core.

\textbf{Definition:} Removal of two or more flakes struck into the face
of a core across the platform, forming a platform with two or more
parallel facets.

\textbf{Inclusion criteria:} Requires hierarchical core reduction.
Described in text OR shown in illustration.

\textbf{Exclusion criteria:} Absence of hierarchical cores. Not
described in text AND not shown in figures.

\textbf{Typical exemplars:} Code if facetting of platforms is described
in the text as a method of preparing platforms. Code if illustrations
show evidence of faceting and if the faceting flakes were likely struck
into the face of the core. Figure 1. Pieces a, c, d, e, f, g, h.

\textbf{Atypical exemplars:} Figure 1. Piece b. Figure 2. Pieces 1, 3-7,
Figure 3. Step D and flake illustration in step E.

\textbf{Close but no:}\\
Phrases like ``platforms were carefully prepared'' without additional
supporting information about the nature of that preparation. Single or
very sparse instances of facetted or dihedral platforms:
\emph{``Nevertheless it is important to mention that within the excluded
material of the `MSA base-complex' (see above) one flake\ldots{} shows
characteristics of a Levallois preferential flake with centripetal
dorsal scars and a facetted striking platform\ldots{}'' (Schmidt, 2011,
p.~92)}. Determining presence based on figures alone requires figures of
the platforms themselves. Do not code based on figures showing only
dorsal or ventral views of flakes. All examples in figure 4 would be
insufficient to code as present.

\hypertarget{face-shaping-through-radial-removals}{%
\subsection{Face shaping through radial
removals}\label{face-shaping-through-radial-removals}}

\textbf{Short Description:} Shaping of the face of a core through
centripetal removals along the perimeter of a core face.

\textbf{Definition:} Flakes taken to modify the distal, lateral,
proximal convexities of a core face, to prepare it for preferential
removals. The preferential removals could be unidirectional or
bidirectional blades, or preferential flakes.

\textbf{Inclusion criteria:} Requires hierarchical core reduction.
Described in text OR shown in figures.

\textbf{Exclusion criteria:} Not described in text AND not shown in
figures.

\textbf{Typical exemplars:} Figure 5. Pieces 1, 2, 4, 7. Consistent
evidence in figures of radial scars on blanks along with cores with
evidence of centripetal preparation. Core faces should be rounded/oval,
not rectilinear. Description of centripetal preparation of cores.

\textbf{Atypical exemplars:} Cases where core faces are described as
shaped through lateral removals, and removals from the distal margin,
but which fit this definition of centripetal flaking.

\textbf{Close but no:}\\
Figure 5. Pieces 3, 5, 6. Cores of a rectilinear shape with lateral
trimming, and distal trimming.

\hypertarget{lateral-trimming}{%
\subsection{Lateral trimming}\label{lateral-trimming}}

\textbf{Short Description:} Lateral convexity of core face is shaped
with flakes initiated from platforms along lateral margins of core.

\textbf{Definition} The lateral convexities of the face of a core are
trimmed through the removal of flakes from platforms roughly parallel to
main flaking axis.

\textbf{Inclusion criteria:} Requires hierarchical core reduction.
Described in text OR shown in figures.

\textbf{Exclusion criteria:} Absence of hierarchical core reduction. Not
described in text AND not shown in figures.

\textbf{Typical exemplars:} Figure 6, no. 4. Descriptions of shaping of
lateral convexities of the face through removals from the lateral margin
of the core.

\textbf{Atypical exemplars:}\\
Figure 5. Piece 3.

\textbf{Close but no:} Descriptions of \emph{débordante} removals, but
without further specifics about the orientation of trimming flakes. Do
not count if: lateral flakes were struck only during radial/centripetal
preparation of a core face, lateral convexities are only trimmed through
\emph{débordante} removals meeting this codebook's definition, the
lateral margin of a core has flake scars parallel to the main flaking
axis, but those flakes originated from a crest used to establish the
core face.

\hypertarget{distal-trimming}{%
\subsection{Distal trimming}\label{distal-trimming}}

\textbf{Short Description:} Face shaped with flakes initiated from
platform at distal margin of core.

\textbf{Definition:} Distal convexities of the face of a hierarchical
core are managed through the removal of flakes from a platform at the
distal margin of the core (i.e.~the platform is perpendicular to the
main flaking axis, and roughly parallel to the main platform).

\textbf{Inclusion criteria:} Requires hierarchical core reduction.
Described in text OR shown in figures.

\textbf{Exclusion criteria:} Absence of hierarchical core reduction. Not
described in text AND not shown in figures.

\textbf{Typical exemplars:} Figure 5, Levallois cores numbers 5, 6.,
Figure 6, Levallois core with lateral trimming, number 4. Figure 7. Step
4 in blademaking sequence. Descriptions of shaping of distal convexities
of the face through removals from the distal margin of the core.

\textbf{Atypical exemplars:} Figure 8. Step 3 in Chazan's description of
La Ferrassie bladelets. Notching a bladelet, to supply end point of
microblades taken from lateral margin of that bladelet core. Figure 10,
numbers 4 and 7. Any flaking at a distal platform that is argued to
control length of flaking surface or to shape the distal end of the
blank.

\textbf{Close but no:} Do not count if: the core could be reasonably
characterized as centripetally prepared, or if these are flakes
initiated from the distal margin that trim the lateral convexities of a
core, like what we might find in a Nubian Levallois core (these should
be coded as \emph{débordante}).

\hypertarget{back-shaping}{%
\subsection{Back shaping}\label{back-shaping}}

\textbf{Short Description:} Back of the core is shaped.

\textbf{Definition:} Back of core is shaped, as is case among naviform
cores, and often among microblade cores. All examples so far identified
are cases where a nodule was bifacially flaked. One of the flaked crests
then is used to remove one or two crested blades to establish a platform
and face. Then flakes are removed from that platfom until exhausted.
When exhausted there remains evidence of original bifacial flaking at
the back of the core.

\textbf{Inclusion criteria:} Shown in figures.

\textbf{Exclusion criteria:} Not shown in figures.

\textbf{Typical exemplars:} Figure 9. Piece 2b. Back of hierarchical
core is shaped, typically bifacially.

\textbf{Atypical exemplars:} NA

\textbf{Close but no:}\\
Crested blades are present, but no illustrations of cores showing
evidence for a modified back/non-flaking surface.

\hypertarget{cresting}{%
\subsection{Cresting}\label{cresting}}

\textbf{Short Description:} Cresting to shape core face during initial
steps of core preparation.

\textbf{Definition:} A core is bifacially or unifacially flaked along
one axis. That crest establishes an artificial ridge along which an
elongated flake, with a crested (entirely or partially) dorsal surface
is removed.

\textbf{Inclusion criteria:} Described in text OR shown in figures.

\textbf{Exclusion criteria:} Not described in text AND not shown in
figures.

\textbf{Typical exemplars:} Figure 10. Flakes whose dorsal surface bears
a full bifacially flaked crest. Pieces 1, 2, 4, 5, 6, 7. Cresting of
core faces is described as present and/or figures show elements flakes
bearing a crest, with at least a partial crest platform present.

\textbf{Atypical exemplars:} The below examples are enough to count
cresting as present: \emph{``The rarity of crested blades (0.3\%, Table
7) and the presence of blades with centripetal flake dorsal scars on one
side only (4.2\%, Fig. 9e,h) are most consistent with a reduction
strategy that generally prepared the blade exploitation surface with
centripetal flake removals''} (Wilkins \& Chazan, 2012 p.~10).
\emph{``Crested elements (n=26) provide some details of the methods used
for initial core preparation. The relatively low number of these pieces
compared to the high number of cores used to produce blade/bladelet
suggests that, in general, cresting was not necessary, and the natural
shape of cobbles/nodules allowed the production of elongated debitage
without cresting.''} (Smith et al., 2016).

\textbf{Close but no:} Flakes with laterally oriented dorsal scars.
Partially crested blades, where no element of the platform of the crest
is present. Ski spall flakes removed during reduction of naviform cores
which were prepared through cresting. Flakes with unifacial cresting,
either complete or partial as in the case of striking platform removals
as described in Smith et al.~2016, which were not argued to play a role
in initial core shaping. Example 3 in Figure 10.

\hypertarget{duxe9bordante}{%
\subsection{\texorpdfstring{\emph{Débordante}}{Débordante}}\label{duxe9bordante}}

\textbf{Short Description:} Shaping of core face through knapping
elongated flakes along lateral margins of core face, across the axis of
the main flaking surface.

\textbf{Definition:} Elongated flakes modify the lateral convexities of
a core face, knapped along the axis of the main flaking surface.
Débordante flakes/blades only include materials that maintain lateral
and distal and sometimes proximal convexities of a core face. These can
be removed either from the proximal/main core platform (more common), or
distal area of the core (see atypical exemplars below). These tend to
have a wedge shaped cross section. In near eastern traditions these are
sometimes refered to as naturally backed blades or knives (Shimelmitz et
al., 2011; Smith et al., 2016) though these sometimes might refer to
elements that are not from hierarchical preferential cores, that happen
to have a wedge shaped cross section.

\textbf{Inclusion criteria:} Requires hierarchical core reduction.
Described in text OR shown in figures.

\textbf{Exclusion criteria:} Not described in text AND not shown in
figures.

\textbf{Typical exemplars:} The term debordante is used and likely is
not referring to what we would otherwise code as lateral trimming.

From Wilkins and Chazan** ``\emph{Débordante} flakes/blades with
unidirectional or bidirectional dorsal scars and a preserved lateral
platform surface are also present in the assemblage (n ¼ 13, 1.3\%, Fig.
9a) and may have sometimes been used to rejuvenate core edges''(Wilkins
\& Chazan, 2012). Figure 5. Pieces 5, 6, and 9 in figure below.

\textbf{Atypical exemplars:} If backed flake/knife is used, only count
as present if there is further discussion about the role they play in
managing convexities of core face, OR if core faces show clear evidence
of \emph{débordante} removals in illustrations. For example \emph{``Only
one elongated flake could be attributed to the Taramsa reduction method.
The presence of many backed pieces (N=10) confirms this trend, since
these flakes are removed to maintain the strong lateral convexity of the
core''(Spinapolice \& Garcea, 2013)}. If ``debordante'' is not used, but
there are phrases like ``elongated flakes were used to align the core
face/modify lateral and distal convexities/modify core edges'' count as
present. Figure 5. Piece 8.

\textbf{Close but no:} Descriptions of \emph{débordante} removals, but
without further specifics about the orientation of trimming flakes
either in text or in figures. Discussion of face shaping with flake
removals or discussion of naturally backed blades/knives/flakes without
mention of the role they served in maintaining face shape/convexity.

\hypertarget{overshot-flakes}{%
\subsection{Overshot flakes}\label{overshot-flakes}}

\textbf{Short Description:} Elongated flake removals that clip or remove
the distal margin of the core.

\textbf{Definition:} Medial and distal convexities of a core face on a
preferential hierarchical core are modified with an invasive flake that
removes the distal end of the core, which may bear a platform (often
opposed to its own) on the distal margin.

\textbf{Inclusion criteria:} Requires hierarchical core reduction.
Described in text OR shown in figures.

\textbf{Exclusion criteria:} Absence of hierarchical core reduction. Not
described in text AND not shown in figures. No argument for presence. No
Hierarchical preferential cores. No overshot blades/flakes in
illustrations. No discussion of overshot blade/flake role in modifying
distal convexity/rejuvenating distal platform.

\textbf{Typical exemplars:} Shimelmitz et al.~2011 \emph{``The frequent
removal of laminar items with an overpassing end termination along the
reduction in order to control core convexities.''}

\textbf{Atypical exemplars:} From Wilkins and Chazan. \emph{``Blades
preserving a distal striking platform (Fig. 7d) further attest to the
bidirectional production of blades.''}. This phrase has enough
information for us to code this as an overshot blade used to manage
distal convexities, but there is also supporting information in the
figure showing overshot blades.

\textbf{Close but no:} Morphological overshot pieces are present, but
there is no discussed role in modifying distal convexity/rejuvenating a
distal platform. This will most often be the case where there are
overshot flakes taken from cores that do not have another platfrom at
the distal end of the core face, or where the overshot flake on a single
platform core does not remove stacking towards the distal end of the
core face. Flakes that have a distal end of core, but the distal end is
cortex.

\hypertarget{kombewa}{%
\subsection{Kombewa}\label{kombewa}}

\textbf{Short Description:} Removal of flake from ventral surface of a
flake

\textbf{Definition:} Ventral surface of a flake is treated as a core
face, where the original flake platform is treated as the core platform.

\textbf{Inclusion criteria:} Described in text OR shown in figures.

\textbf{Exclusion criteria:} Not described in text AND not shown in
figures.

\textbf{Typical exemplars:} Kombewa technique described as present,
``Janus flakes'' and the processes to make them are described.
``Cleavers were made on sometimes completely unmodified Kombewa flakes''

\textbf{Atypical exemplars:} NA

\textbf{Close but no:} Core on flake described as present without
discussion of where the flaking surface is on that flake. Flakes from
retouching large flakes, which are initiated at dorsal margin, and
capture some of the ventral surface, but were not initiated from the
original platform, and did not remove much of the ventral surface. Scars
not propagated across ventral surface. Burin spalls, or \emph{tranchet}
spalls taken from flakes. Kombewa flakes should not have two ventral
surfaces.

\hypertarget{core-tablet-removals}{%
\subsection{Core tablet removals}\label{core-tablet-removals}}

\textbf{Short Description:} Flake removals that rejuvinate or prepare a
core platform, by removing some or all of the core platform.

\textbf{Definition:} Striking a flake into the face of a core, where the
dorsal surface of that flake is the main platform of a core. These are
intended to rejuvinate the core platform by establishing a fresh flaking
surface.

\textbf{Inclusion criteria:} Described in text OR shown in figures.

\textbf{Exclusion criteria:} Not described in text AND not shown in
figures.

\textbf{Typical exemplars:} Figure 11. Cores a and d.~Descriptions of
core tablets, or illustrations of core tablets themselves and text
describing the function they served in rejuvenating core platforms, or
illustrations of cores with strong evidence for core tablet removals.
Phrases like,`the assemblage has several core tablets', or ``core
platforms were rejuvenated through striking core tablets''.

\textbf{Atypical exemplars:} Figure 11. Cores b and e show evidence of
flakes taken across the top of the core (the negative bulb of the flake
is present) and this flake erased the negative bulbs of several of the
blade scars (the flake did not just establish a platform, but was taken
after several blades had been struck). These lines of evidence on a
single platform hierarchical core are sufficient to infer that these
cores were rejuvenated with tablets. Figure 12. Pieces 1-4 show
insufficient evidence to call them core tablets, but the presence of
additional text describing their role in rejuvenating platforms gives us
enough infromation to call them core tablets.

\textbf{Close but no:} Coding as present should not be based only on
illustrations of pieces that look like tablets. It should also not
normally be based on illustrations of cores that have a single scar as
the platform without supporting evidence that that scar was made to
rejuvenate the platfrom. Flakes with large facetted proximal margins are
illustrated without any other information in the text about rejuvenation
of platforms. Figure 12. Pieces 1-4. Without additional information
about the geometry of the core from which these were taken, we should
not call these core tablets (though the paper from which the figure is
borrowed provides enough context to call these tablets).

\hypertarget{abrasiongrinding}{%
\subsection{Abrasion/grinding}\label{abrasiongrinding}}

\textbf{Short Description:} Abrasion or grinding performed at any point
in reduction sequence.

\textbf{Definition:} Core was abraded/ground to strengthen platform, or
tools were abraded or ground as part of production sequence.

\textbf{Inclusion criteria:} Described in text OR shown in figures.

\textbf{Exclusion criteria:} Not described in text AND not shown in
figures.

\textbf{Typical exemplars:} Any discussion of the abrasion of the core
platform in relation to platform preparation in the text, or the
presence of grinding/abrasion to finish a tool. For example: `platforms
were prepared by abrasion.'. Figures showing evidence of abrasion, or
polishing of platforms or of edges. Presence of polished adzes, and
groundstone tools that are not formed only through pecking.

\textbf{Atypical exemplars:} ``These bifaces have several common
attributes. All were primarily shaped by abrasion by rubbing with a
coarse material, leaving parallel striae on their surface'' (Rosenthal,
1996).

\textbf{Close but no:} Platform preparation described, but the kind of
preparation not explicitly described. Tools show suggestive, though
ambigious evidence of being ground/abraded but no discussion of the
technique in text. Presence of use polish (e.g.~sickle polish). Figure
13. shows a hammer dressed/pecked tool, which may have been abraded too,
but without additional info in text or in other figures do not call it
present.

\hypertarget{overhang-removalmicrochipping-of-area-below-platform}{%
\subsection{Overhang removal/microchipping of area below
platform}\label{overhang-removalmicrochipping-of-area-below-platform}}

\textbf{Short Description:} Removal of chips to modify area below
platform.

\textbf{Definition:} Removal of chips initiated from platform to modify
proximal margin of core face/proximal convexities, and modify the
platform angle.

\textbf{Inclusion criteria:} Described in text OR shown in figures.

\textbf{Exclusion criteria:} Not described in text AND not shown in
figures.

\textbf{Typical exemplars:} Illustrations show small flakes removed at
proximal margin of flakes, or on areas below core platforms.
Descriptions of overhang removal, or microchipping of platform in text.

\textbf{Atypical exemplars:} NA

\textbf{Close but no:} Similar practices described in text, but this
refers to what we would otherwise code as faceting.

\hypertarget{percussion-by-striking-with-hard-hammer}{%
\subsection{Percussion by striking with hard
hammer}\label{percussion-by-striking-with-hard-hammer}}

\textbf{Short Description:} Use of a hard hammer

\textbf{Definition:} Use of a hard hammer to strike onto some substrate,
whether it is a core, held in the hand, mounted on an anvil, or whether
the core itself was struck on a hard substrate (as in case of passive
hammer technique). Flakes need not to have been produced.

\textbf{Inclusion criteria:} Flakes produced OR hard hammers present.

\textbf{Exclusion criteria:} (No flakes produced AND no hard hammers
present) OR described as absent.

\textbf{Typical exemplars:} If flakes are produced in assemblage,but
there is no discussion about the kind of hammer/percussion technique
used, count as present.

\textbf{Atypical exemplars:} Examples of technologies where there is
only percussion, with no production of flakes (e.g.~nutcracking).
Explicit description of hard hammer percussion as method of flaking.
Hammer dressing/pecking as a method of shaping tools.

\textbf{Close but no:} Flakes produced, but only percussion techniques
described do not include use of hard hammer. For example: later blade
production techniques which likely involved hard hammer percussion in
early stages of core preparation, but where only the later stages of
blank production are represented and these are described as involving
soft hammer/indirect percussion/pressure flaking.

\hypertarget{core-supported-by-hand}{%
\subsection{Core supported by hand}\label{core-supported-by-hand}}

\textbf{Short Description:} Any stage of tool manufacture includes
holding the core in hand while striking it (i.e.~use of anvil is
absent).

\textbf{Definition:} NA

\textbf{Inclusion criteria:} Flakes produced OR any hard hammer,soft
hammer,or indirect percussion present.

\textbf{Exclusion criteria:} Bipolar percussion is main mode of flaking
OR passive hammer is main mode of flaking OR cores described as
primarily mechanically mounted.

\textbf{Typical exemplars:} ``Nodules were reduced through freehand
percussion''.

\textbf{Atypical exemplars:} ``Early stage reduction likely included
soft hammer percussion and hard hammer percussion to shape core prior to
it being reduced through bipolar percussion'', ``Nodules were bifacially
flaked to produce a crest, prior to being mounted in a substrate where
flaking was then performed through application of pressure''.

\textbf{Close but no:} Passive hammer percussion resulting in flakes.

\hypertarget{use-of-anvil}{%
\subsection{Use of anvil}\label{use-of-anvil}}

\textbf{Short Description:} Any incorporation of an anvil, or hard
substrate, in the reduction process.

\textbf{Definition:} Use of an anvil or hard substrate at any point in
the tool reduction sequence.

\textbf{Inclusion criteria:} Described in text OR shown in figures.

\textbf{Exclusion criteria:} Not described in text AND not shown in
figures.

\textbf{Typical exemplars:} Any discussion of anvil use, including
bipolar percussion.

\textbf{Atypical exemplars:} Descriptions or illustrations of anvil
resting: \emph{``\ldots many of our early replications were performed
with the aid of an anvil (Figure 5), and this was found to be a
successful technique for creating the initial steep sides on large
flakes and cobbles. Anvil resting was more successful than true bipolar
flaking in generating steep-angled edges and moving flaking on to new
edges''} (Clarkson et al., 2015 p.74)

Presence of \emph{pièces esquillées} or other kinds of scaled pieces
described as resulting from smashing a flake against a hard substrate.

\textbf{Close but no:} Cases where bipolar is mentioned briefly, but
where there is a possiblity that it refers to bidirectional flaking. In
such cases, there must also be some visual evidence for bipolar
percussion with an anvil (in the form of scaled pieces, for example).

\hypertarget{core-rotation}{%
\subsection{Core rotation}\label{core-rotation}}

\textbf{Definition:} Core rotated at any point in reduction sequence

\textbf{Inclusion criteria:} Described in text OR shown in figures.

\textbf{Exclusion criteria:} Not described in text AND not shown in
figures.

\textbf{Typical exemplars:} Figures showing cores with flake removals
across two different axes. Bifacial flaking of any kind.

\textbf{Atypical exemplars:} NA

\textbf{Close but no:} Single platform cores are present without
evidence of removals across the top of the core.

\hypertarget{soft-hammer}{%
\subsection{Soft Hammer}\label{soft-hammer}}

\textbf{Short Description:} Use of a soft hammer

\textbf{Definition:} Use of a soft hammer (whether the material be wood,
bone, soft stone, etc).

\textbf{Inclusion criteria:} Described in text OR shown in figures.

\textbf{Exclusion criteria:} Not described in text AND not shown in
figures.

\textbf{Typical exemplars:} Authors explicitly state that soft hammer
percussion was likely performed, soft hammers present in the
archaeological assemblage and flakes present consistent with use of soft
hammers. For example, if delicate, thin, wide flakes are present and
soft-hammers were recovered from related contexts.

\textbf{Atypical exemplars:} Examples where indirect percussion, soft
hammer, and pressure flaking are all reported as possibilities, but soft
hammer percussion is reported as most likely.

\textbf{Close but no:} Soft hammer use not specified, thinned bifaces
appear consistent with soft-hammer percussion but no description in
text. Flakes with lipped platforms present, but no discussion about
soft-hammer percussion. Statements like ``soft hammer percussion is one
method that could produce the forms here'', without noting that it is
the most likely method.

\hypertarget{indirect-percussion}{%
\subsection{Indirect percussion}\label{indirect-percussion}}

\textbf{Short Description:} Use of a punch to remove flakes

\textbf{Definition:} Use of a punch of any given material placed on a
platform, and struck with a hammer to punch flakes from the core.

\textbf{Inclusion criteria:} Described in text

\textbf{Exclusion criteria:} Not described in text.

\textbf{Typical exemplars:} `The blades in this assemblages would have
likely required indirect percussion', `experimental reconstructions of
the stitching pattern on these Danish daggers indicate that indirect
percussion/use of a punch would have been required'. Examples where
indirect percussion, soft hammer, and pressure flaking are all reported
as possibilities, but indirect percussion is reported as most likely.

\textbf{Atypical exemplars:} NA

\textbf{Close but no:} `Indirect percussion could have been employed',
figures showing evidence of stitching patterns, or flaking patterns that
exploit pronounced negative bulbs of percussion as platforms that appear
difficult to achieve without indirect percussion, but could potentially
have been produced through pressure flaking. Statements like ``indirect
percussion is one method that could produce the forms here'', without
noting that it is the most likely method.

\hypertarget{flaking-with-application-of-pressure}{%
\subsection{Flaking with application of
pressure}\label{flaking-with-application-of-pressure}}

\textbf{Short Description:} Removal of flakes through application of
pressure on core platform, or on a retouched tool edge.

\textbf{Definition:} Use of typically soft indentor, bone, metal, or
hard wood,to press off flakes, or burin spalls.

\textbf{Inclusion criteria:} Described in text OR shown in figures.

\textbf{Exclusion criteria:} Not described in text AND not shown in
figures.

\textbf{Typical exemplars:} Illustrations of pressure flaked bifaces, or
of retouch through pressure flaking. These show invasive flake scars on
retouched pieces that are extremely narrow (\textasciitilde.5mm), thin
(\textless.1mm, and relatively invasive (\textasciitilde5mm).

\textbf{Atypical exemplars:} `Microblades were struck through
application of pressure', 'burin spalls were likely removed through
pressure flaking''. Examples where indirect percussion, soft hammer, and
pressure flaking are all reported as possibilities, but pressure flaking
is reported as most likely.

\textbf{Close but no:} Delicate retouched tools without strong evidence
of pressure flaking retouch. Descriptions of delicate blade manufacture,
but no clear statement indicating that pressure flaking was the likely
method of core reduction. Statements like ``pressure flaking is one
method that could produce the forms here'', without noting that it is
the most likely method.

\hypertarget{peckinghammer-dressing}{%
\subsection{Pecking/hammer dressing}\label{peckinghammer-dressing}}

\textbf{Short Description:} Modification of core or tool through pecking

\textbf{Definition:} NA

\textbf{Inclusion criteria:} Described in text OR shown in figures.

\textbf{Exclusion criteria:} Not described in text AND not shown in
figures.

\textbf{Typical exemplars:} Figure 13. Hammer dressing, or pecking
described in text as method employed at any point in tool manufacture.
Hammer dressing is unambiguously present in illustrations.

\textbf{Atypical exemplars:} NA

\textbf{Close but no:} Artifacts appear pecked in figures but this could
be consistent with either use, or post-depositional processes and there
is no clarification in text.

\hypertarget{invasive-flaking}{%
\subsection{Invasive flaking}\label{invasive-flaking}}

\textbf{Short Description:} Removal of non-cortical flakes that extend
beyond the midpoint of the core face, or beyond the midpoint of a flake
or core-tool during retouch.

\textbf{Definition:} Removal of non-cortical flakes that extend beyond
the midpoint of the core face, or beyond the midpoint of a flake or
core-tool during retouch. The midpoint can either be in relation to the
long axis (if striking things like burin spalls or blades), or along the
short axis (in cases of tranchet resharpening, and biface thinning).

\textbf{Inclusion criteria:} Described in text OR shown in figures.

\textbf{Exclusion criteria:} Not described in text AND not shown in
figures.

\textbf{Typical exemplars:} Examples of blade production, and biface
thinning.

\textbf{Atypical exemplars:} Burin spall removals that extend halfway
down the axis of a flake, tranchet retouch.

\textbf{Close but no:} Invasive flaking described only in text, but
without further information about how invasive the flake are. No
description of flakes invasive to the degree that they extend beyond
midline. Candidate invasive flakes are cortical.

\hypertarget{ochre-use}{%
\subsection{Ochre use}\label{ochre-use}}

\textbf{Short Description:} Use of ochre in any stage of tool making

\textbf{Definition:}Use of ochre as a pigment or as a binding agent.

\textbf{Inclusion criteria:} Described in text.

\textbf{Exclusion criteria:} Not described in text.

\textbf{Typical exemplars:} `Ochre was applied to points', `the adhesive
residues include ochre'

\textbf{Atypical exemplars:} NA

\textbf{Close but no:} NA

\hypertarget{asphalt-use}{%
\subsection{Asphalt use}\label{asphalt-use}}

\textbf{Short Description:} Use of aslphalt at any stage of the tool
making process.

\textbf{Definition:} Use of asphalt as a binding agent

\textbf{Inclusion criteria:} Described in text.

\textbf{Exclusion criteria:} Not described in text.

\textbf{Typical exemplars:} `asphalt was applied to points', `the
adhesive residues include asphalt'. Asphalt adhered to tool, typically
at its base/tang.

\textbf{Atypical exemplars:} NA

\textbf{Close but no:} Asphalt in association with tools likely through
post-depositional processes.

\hypertarget{tanging}{%
\subsection{Tanging}\label{tanging}}

\textbf{Short Description:} Retouching base of piece to form a tang

\textbf{Definition:} Retouching a piece, typically through backing and
notching to aid with hafting.

\textbf{Inclusion criteria:} Described in text OR shown in figures.

\textbf{Exclusion criteria:} Not described in text AND not shown in
figures.

\textbf{Typical exemplars:} Stemmed points, side notched points (Figure
14, all points shown). Description of retouch as forming a tang for the
purpose of hafting.

\textbf{Atypical exemplars:} No mention of tanging, but pieces with
basal modifications consistent with a tang are described as frequent or
have many illustrations. Tangs associated with pieces like Aterian
points.

\textbf{Close but no:} Pieces likely hafted, but where the only basal
modification is a truncation/backing, such as atypical salabiya points
discussed in Smith et al.~2016. Pieces with ``tangs'' that are
interpreted as drills and perforators.

\hypertarget{retouch}{%
\subsection{Retouch}\label{retouch}}

\textbf{Short Description:} Retouch of flake

\textbf{Definition:} Retouch of any kind present. Small flakes are
removed from either a retouched flake/blank or core tool to shape it or
resharpen it.

\textbf{Inclusion criteria:} Retouch described as present. Illustrations
of pieces show unambiguous evidence of retouch.

\textbf{Exclusion criteria:} Bifacial retouch is present. No unambiguous
illustrations of unifacially retouched pieces, and no mention of
retouch.

\textbf{Typical exemplars:} NA

\textbf{Atypical exemplars:} NA

\textbf{Close but no:} NA

\hypertarget{abrupt-retouch.}{%
\subsection{Abrupt retouch.}\label{abrupt-retouch.}}

\textbf{Short Description:} Retouch forms an abrupt, scraper-like margin

\textbf{Definition:} Retouch that increases the angle of the margin to
\textasciitilde70-90 degrees.

\textbf{Inclusion criteria:} Described in text OR shown in figures.

\textbf{Exclusion criteria:} Not described in text AND not shown in
figures.

\textbf{Typical exemplars:} Presence of artifacts with retouch that
forms an angle greater than 70 degrees. Description of scrapers in
assemblage.

\textbf{Atypical exemplars:} Tanged pieces where tangs are formed by
lengths of abrupt retouch.

\textbf{Close but no:} Abrupt retouch on flat thin flakes. Cases that
are best described as notches or denticulates. Retouch is abundant, but
but lack of evidence for artifacts with retouch forming greater than 70
degree angle.

\hypertarget{notching}{%
\subsection{Notching}\label{notching}}

\textbf{Short Description:} Retouch forms round concavity.

\textbf{Definition:} Retouch, either unifacial or bifacial, forms a
round concavity, or series of concavities.

\textbf{Inclusion criteria:} Described in text OR shown in figures.

\textbf{Exclusion criteria:} Not described in text AND not shown in
figures.

\textbf{Typical exemplars:} Figure 14. Piece 1. Figure 15. Pieces E, F.
Description of notches, or denticulates. Illustration of pieces with
notches formed.

\textbf{Atypical exemplars:} Step 3 in Figure 8. Lateral notches on
artifact 1 in Figure 14. The complete ``ear'' on artifact 2 in Figure
14.

\textbf{Close but no:} Basal concavities on points. Basal concavity on
artifact 1 in Figure 14.

\hypertarget{burination}{%
\subsection{Burination}\label{burination}}

\textbf{Short Description:} Flaking of burin spalls.

\textbf{Definition:} Removal of flakes along the margin of another
flake. Flakes from this process should have two ventral surfaces, the
parent flake's, and its own.

\textbf{Inclusion criteria:} Described in text OR shown in figures.

\textbf{Exclusion criteria:} Not described in text AND not shown in
figures.

\textbf{Typical exemplars:} Figure 16. from Smith et al.~2016. all
examples. Illustrations of flakes with evidence of spalls taking across
their lateral, proximal, or distal margins. Burins, or burin spalls
described as present.

\textbf{Atypical exemplars:} Steps 2,4, and 5 in Figure 8. Systematic
production of Krukowski microburins and the regular microburin technique
are both examples of burination, as the resulting flakes have two
ventral surfaces. However, isolated instances should not be coded as
these can be produced by accident (de Wilde and de Bie 2011). The only
spalls meeting the definition produced through bipolar percussion should
be pieces like microburins or Krukowski microburins.

\textbf{Close but no:} Mention of `impact burination', `spalling',
`core-on-flake'. Burin spall-like removals on bifaces, cores, or core
tools, which are better characterized as tranchet spalls, or crested
blades. Spalls from scaled pieces, scaled pieces themselves or
\emph{pièces esquillées}. Isolated examples of Krukowski microburins and
regular microburins.

\hypertarget{tranchet-removal.}{%
\subsection{Tranchet removal.}\label{tranchet-removal.}}

\textbf{Short Description:} Retouch of a core-tool by removing a flake
across the face at the distal margin or bit.

\textbf{Definition:} Retouch of a core-tool by removing a flake across
the face at the distal margin or bit. The spall removed in this process
is typically curved, and may be trihedral in a way similar to a burin
spall. One lateral margin of the spall will be the distal margin/bit of
the core-tool. The opposite face of the core tool will have a remnant on
one face of the spall, adjacent to the ventral surface of the spall.

\textbf{Inclusion criteria:} Described in text OR shown in figures.

\textbf{Exclusion criteria:} Not described in text AND not shown in
figures.

\textbf{Typical exemplars:} Description of tranchet resharpening in
text, or \emph{coup de tranchet}. Illustration of core-tools with
negative from tranchet spall visible: for example in Figure 17. Pieces
1-3. Illustration of tranchet spalls themselves with further description
that role they played in shaping core tools.

\textbf{Atypical exemplars:} Illustrations of multiple core-tools with
examples of tranchet resharpening, but no description of this method of
resharpening in text. Tranchet axes on flakes with predominately
unifacial retouch, whose tranchet spalls could be reasonably interpreted
as burins (code them as tranchet spalls).

\textbf{Close but no:} Illustrations of what appear to be tranchet
spalls in context of a site with core-tools that could reasonably have
been resharpened with such spalls, but no description in the text that
these spalls served that purpose. Examples of micro ``tranchet'' adzes
in the Solomon Islands illustrated in Harrison (1931) which have the
same kind of bevelled bit as a tranchet axe, but which were likely
produced through careful preparation of a core, rather than through a
\emph{coup de tranchet}.

\hypertarget{figures}{%
\section{Figures}\label{figures}}

Figure 1. Figure 23 illustrating levallois point variation from Kibish
formation (Shea, 2008).

Figure 2. From figure 5 illustrating levallois flake and centripetal
flake diversity (Picin \& Vaquero, 2016).

Figure 3. Figure 2 illustrating schematic drawings of blade manufacture
methods in Queensland (Moore, 2003).

Figure 4. Figure 4 on technological blade classifications at Rose
Cottage Cave (Soriano et al., 2007).

Figure 5. Figure 1 in in description of Levallois technology (Bordes,
1980).

Figure 6. Figure 2 in description of Levallois technology (Bordes,
1980).

Figure 7. Figure 5 description of bladelet core preparation at 'Ein
Qashish with distal preparation at step 4.(Malinsky-Buller et al., 2014)

Figure 8. Figure 2. Schematic illustrating busqued burin production
methods at La Ferrassie (Chazan, 2001).

Figure 9. Figure 2. Microblade core variability at Amakomanak.(Coutouly,
2017)

Figure 10. Initial blade subtypes from Kfar HaHoresh (Barzilai \&
Goring-Morris, 2010).

Figure 11. Figure 4 in description of blade cores from Fumane cave
(Falcucci \& Peresani, 2018).

Figure 12. Figure 14 illustrating platform spalls from bidirectional
blade cores recovered from Kfar HaHoresh (Barzilai \& Goring-Morris,
2010)

Figure 13. Figure 2 illustrating hammer dressing on stemmed obsidian
tool from Biak Island, West Papua (Robin Torrence et al., 2009).

Figure 14. Figure 13 illustrating projectile points recovered from Motza
(Khalaily et al., 2007).

Figure 15. Figure 8 illustrating some retouched tool tyles from Ayn Abu
Nukhayla (Henry \& Mraz, 2020).

Figure 16. Figure 5 illustrating burin variation at the PPNA site El
Hemmeh (Smith et al., 2016).

Figure 17. Figure 14 illustrating tranchet axe variability at Motza
(Khalaily et al., 2007).

\hypertarget{bibliography}{%
\section{Bibliography}\label{bibliography}}

Barzilai, O., \& Goring-Morris, A. N. (2010). Bidirectional Blade
Production at the PPNB Site of Kfar HaHoresh: The Techno-Typological
Analysis of a Workshop Dump. Paléorient, 36(2), 5--34.
\url{https://doi.org/10.3406/paleo.2010.5386}

Bordes, F. (1980). Le débitage Levallois et ses variantes. Bulletin de
la Société préhistorique française, 77(2), 45--49.
\url{https://doi.org/10.3406/bspf.1980.5242}

Chazan, M. (2001). Bladelet Production in the Aurignacian of La
Ferrassie (Dordogne, France). Lithic Technology, 26(1), 16--28.
\url{https://doi.org/10.1080/01977261.2001.11720973}

Clarkson, C., Shipton, C., \& Weisler, M. (2015). Front, back and sides:
Experimental replication and archaeological analysis of Hawaiian adzes
and associated debitage: A study of Hawaiian adze manufacture.
Archaeology in Oceania, 50(2), 71--84.
\url{https://doi.org/10.1002/arco.5056}

Coutouly, Y. A. G. (2017). Amakomanak: An Early Holocene Microblade Site
in Northwestern Alaska. Arctic Anthropology, 54(2), 111--135.
\url{https://doi.org/10.3368/aa.54.2.111}

De Wilde, D., \& De Bie, M. (2011). On the origin and significance of
microburins: an experimental approach. Antiquity, 85(329), 729-741.

Falcucci, A., \& Peresani, M. (2018). Protoaurignacian Core Reduction
Procedures: Blade and Bladelet Technologies at Fumane Cave. Lithic
Technology, 43(2), 125--140.
\url{https://doi.org/10.1080/01977261.2018.1439681}

Harrison, H. S. (1931). Flint Tranchets in the Solomon Islands and
Elsewhere. The Journal of the Royal Anthropological Institute of Great
Britain and Ireland, 61, 425--434. JSTOR.
\url{https://doi.org/10.2307/2843929}

Henry, D. O., \& Mraz, V. (2020). Lithic economy and prehistoric human
behavioral ecology viewed from southern Jordan. Journal of
Archaeological Science: Reports, 29, 102089.
\url{https://doi.org/10.1016/j.jasrep.2019.102089}

Hruschka, D. J., Schwartz, D., St.John, D. C., Picone-Decaro, E.,
Jenkins, R. A., \& Carey, J. W. (2004). Reliability in Coding Open-Ended
Data: Lessons Learned from HIV Behavioral Research. Field Methods,
16(3), 307--331. \url{https://doi.org/10.1177/1525822X04266540}

Khalaily, H., Bar-Yosef, O., Boaretto, E., Bocquentin, F., Le Dosseur,
G., Erikh-Rose, A., Goring-Morris, A. N., Greenhut, Z., Marder, O.,
Sapir-Hen, L., \& Yizhaq, M. (2007). Excavations at Motza in the Judean
Hills and the Early Pre-Pottery Neolithic B in the Southern Levant.pdf.
Paleorient, 33(2), 5--37.

MacQueen, K. M., McLellan, E., Kay, K., \& Milstein, B. (1998). Codebook
Development for Team-Based Qualitative Analysis. CAM Journal, 10(2),
31--36. \url{https://doi.org/10.1177/1525822X980100020301}

Malinsky-Buller, A., Ekshtain, R., \& Hovers, E. (2014). Organization of
lithic technology at 'Ein Qashish, a late Middle Paleolithic open-air
site in Israel. Quaternary International, 331, 234--247.
\url{https://doi.org/10.1016/j.quaint.2013.05.004}

Moore, M. W. (2003). Australian Aboriginal Blade Production Methods on
the Georgina River, Camooweal, Queensland. Lithic Technology, 28(1),
35--63. \url{https://doi.org/10.1080/01977261.2003.11721001}

Perreault, C., Brantingham, J., Kuhn, S. L., Wurz, S., \& Gao, X.
(2013). Measuring the Complexity of Lithic Technology. Current
Anthropology, 54(8).

Picin, A., \& Vaquero, M. (2016). Flake productivity in the Levallois
recurrent centripetal and discoid technologies: New insights from
experimental and archaeological lithic series. Journal of Archaeological
Science: Reports, 8, 70--81.
\url{https://doi.org/10.1016/j.jasrep.2016.05.062}

Rosenthal, E. J. (1996). San Nicolas Island Bifaces: A Distinctive Stone
Tool Manufacturing Technique. Journal of California and Great Basin
Archaeology, 18(2).

Schmidt, I. (2011). A Middle Stone Age Assemblage with Discoid Lithic
Technology from Etemba 14, Erongo Mountains, Northern Namibia. Journal
of African Archaeology, 9(1), 85--100.

Shea, J. J. (2008). The Middle Stone Age archaeology of the Lower Omo
Valley Kibish Formation: Excavations, lithic assemblages, and inferred
patterns of early Homo sapiens behavior. Journal of Human Evolution,
55(3), 448--485. \url{https://doi.org/10.1016/j.jhevol.2008.05.014}

Shimelmitz, R., Barkai, R., \& Gopher, A. (2011). Systematic blade
production at late Lower Paleolithic (400--200 kyr) Qesem Cave, Israel.
Journal of Human Evolution, 61(4), 458--479.
\url{https://doi.org/10.1016/j.jhevol.2011.06.003}

Smith, S., Paige, J., \& Makarewicz, C. A. (2016). Further diversity in
the Early Neolithic of the Southern Levant: A first look at the PPNA
chipped stone tool assemblage from el-Hemmeh, Southern Jordan.
Paléorient, 42(1), 7--25. \url{https://doi.org/10.3406/paleo.2016.5691}

Soriano, S., Villa, P., \& Wadley, L. (2007). Blade technology and tool
forms in the Middle Stone Age of South Africa: The Howiesons Poort and
post-Howiesons Poort at Rose Cottage Cave. Journal of Archaeological
Science, 34(5), 681--703.
\url{https://doi.org/10.1016/j.jas.2006.06.017}

Spinapolice, E. E., \& Garcea, E. A. A. (2013). The Aterian from the
Jebel Gharbi (Libya): New Technological Perspectives from North Africa.
African Archaeological Review, 30(2), 169--194.
\url{https://doi.org/10.1007/s10437-013-9135-2}

Tversky, A., \& Kahneman, D. (1974). Judgment under Uncertainty:
Heuristics and Biases. Science, 185(4157), 1124--1131.
\url{https://doi.org/10.1126/science.185.4157.1124}

Wilkins, J., \& Chazan, M. (2012). Blade production \textasciitilde500
thousand years ago at Kathu Pan 1, South Africa support for a multiple
origins hypothesis. Journal of Archaeological Science, 1--18.

\end{document}
